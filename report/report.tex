\documentclass{article}
\usepackage[greek,english]{babel}
\usepackage{geometry}
\usepackage{amsmath}
\usepackage{graphicx}
\usepackage{enumitem}
\usepackage{bm}
\usepackage{float}
\usepackage{accents}
\usepackage{undertilde}
\usepackage{fontspec}

% added this to stop hbadness warning
\hbadness = 5000000

\geometry
{
	headheight = 4ex,
	includehead,
	includefoot,
	paper = a4paper,
	inner = 2.5cm,
	outer = 2.5cm,
	bindingoffset = 0.5cm,
	top = 2cm,
	bottom = 1.5cm
}

\setmainfont{Arial}


\title{\Huge Προγραμματιστική Εργασία} 
\vspace{1cm}
\author {\Large Χαρά Τσίρκα, Πρόδρομος Αβραμίδης, Γεώργιος Γεροντίδης}

\begin{document}

\maketitle
\begin{center}
\vspace{1cm}
\includegraphics[width=0.3\textwidth]{images/uthlogo.png}
\vspace{2cm}
\end{center}
\begin{center}
  \Huge Εξόρυξη δεδομένων \vspace{1cm}

  \Large Εαρινό εξάμηνο 2023-2024 \vspace{1cm}

  \Large Καθηγητής: Βασιλακόπουλος Μιχαήλ
\end{center}

\newpage

\section*{Εισαγωγή}

Η εργασία μας επικεντρώνεται στην πρόβλεψη του κόστους ασφάλισης μηχανοκίνητων οχημάτων. Η ανάλυση αυτή αποτελεί ένα κρίσιμο ζήτημα στον τομέα της ασφάλισης, καθώς επιτρέπει στους ασφαλιστές να προσδιορίζουν με μεγαλύτερη ακρίβεια τα ασφαλιστικά ασφάλιστρα, λαμβάνοντας υπόψη διάφορους παράγοντες που επηρεάζουν το κόστος.
Η διαδικασία της εργασίας ξεκινά με την προ-επεξεργασία των δεδομένων, κατά την οποία πραγματοποιήθηκε εξερευνητική ανάλυση (exploratory analysis) για τον προσδιορισμό των κριτηρίων διαχωρισμού των δεδομένων. Κατά τη διάρκεια αυτής της ανάλυσης, μετρήθηκε ο βαθμός επίδρασης κάθε χαρακτηριστικού (feature) του συνόλου δεδομένων στα αποτελέσματα. Με τη βοήθεια διαγραμμάτων, καταφέραμε να επιλέξουμε τον κατάλληλο διαχωρισμό των δεδομένων για περαιτέρω ανάλυση.
Στη συνέχεια, η εργασία θα προχωρήσει στη δημιουργία και αξιολόγηση των μοντέλων πρόβλεψης, λαμβάνοντας υπόψη την είσοδο των χρηστών, ενώ θα ακολουθήσει η οπτικοποίηση και η αξιολόγηση των αποτελεσμάτων.

\end{document}
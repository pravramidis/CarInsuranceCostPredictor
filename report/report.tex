%  Μεταγλώττιση με XeLaTeX

\documentclass{llncs}
\pagestyle{headings} 
%
\usepackage{makeidx}  % allows for indexgeneration
%
\usepackage{xltxtra}
\usepackage{xgreek}

\usepackage{caption}
\usepackage{subcaption}

\setsansfont{Arial}
\setmonofont{Courier New}
\setmainfont[Mapping=tex-text]{Times New Roman}
% \setmainfont[Mapping=tex-text]{GFS Didot} 
% \setsansfont{GFS Didot}
% \setmonofont{GFS Didot}

\renewcommand{\abstractname}{Περίληψη}
\newtheorem{observation}{Παρατήρηση}
\renewcommand{\keywordname}{\bf Λέξεις Κλειδιά:}
\renewcommand\refname{Αναφορές}
\renewcommand\ackname{Ευχαριστίες}
\renewcommand\andname{και}
\renewcommand\corollaryname{Πόρισμα}
\renewcommand\definitionname{Ορισμός}
\renewcommand\examplename{Παράδειγμα}
\renewcommand\exercisename{Άσκηση}
\renewcommand\figurename{Εικ.}
\renewcommand\lemmaname{Λήμμα}
\renewcommand\proofname{Απόδειξη}
\renewcommand\propositionname{Πρόταση}
\renewcommand\solutionname{Λύση}
\renewcommand\tablename{Πίνακας}
\renewcommand\theoremname{Θεώρημα}

\begin{document}
%
\title{\Huge Προγραμματιστική Εργασία Πρόβλεψη κόστους ασφάλισης οχημάτων}
%
%
\author{\Large Χαρά Τσίρκα \and Πρόδρομος Αβραμίδης \and Γεώργιος Γεροντίδης}
%
%
%
\institute{\email{\{ctsirka, pavramidis, ggerontidis\}@e-ce.uth.gr}\\
8$^{ο}$ εξάμηνο\\
\begin{center}
    \vspace{1cm}
    \includegraphics[width=0.4\textwidth]{uthlogo.png}
    \vspace{1cm}
\end{center}
\Large Τμήμα Ηλεκτρολόγων Μηχανικών \& Μηχανικών Υπολογιστών\\
Πανεπιστήμιο Θεσσαλίας, Βόλος\\
\vspace{1cm}
{\bf \Large Εξόρυξη Δεδομένων 2023-24}\\
\Large Διδάσκον: Μ.Βασιλακόπουλος\\
\vspace{1cm}
Μάιος 2024}

\maketitle


\section{Εισαγωγή}
Η εργασία μας επικεντρώνεται στην πρόβλεψη του κόστους ασφάλισης μηχανοκίνητων οχημάτων. Η ανάλυση αυτή αποτελεί ένα κρίσιμο ζήτημα στον τομέα της ασφάλισης, 
καθώς επιτρέπει στους ασφαλιστές να προσδιορίζουν με μεγαλύτερη ακρίβεια τα ασφαλιστικά ασφάλιστρα, λαμβάνοντας υπόψη διάφορους παράγοντες που επηρεάζουν το κόστος.
Η διαδικασία της εργασίας ξεκινά με την προ-επεξεργασία των δεδομένων, κατά την οποία πραγματοποιήθηκε εξερευνητική ανάλυση (exploratory analysis) για τον προσδιορισμό 
των κριτηρίων διαχωρισμού των δεδομένων. Κατά τη διάρκεια αυτής της ανάλυσης, μετρήθηκε ο βαθμός επίδρασης κάθε χαρακτηριστικού (feature) του συνόλου δεδομένων στα αποτελέσματα. 
Με τη βοήθεια διαγραμμάτων, καταφέραμε να επιλέξουμε τον κατάλληλο διαχωρισμό των δεδομένων για περαιτέρω ανάλυση. Στη συνέχεια, η εργασία θα προχωρήσει στη δημιουργία και 
αξιολόγηση των μοντέλων πρόβλεψης, λαμβάνοντας υπόψη την είσοδο των χρηστών, ενώ θα ακολουθήσει η οπτικοποίηση και η αξιολόγηση των αποτελεσμάτων.

\section{Περιγραφή dataset}
Το dataset το οποίο επιλέξαμε αποτελείται από 30 μεταβλητές (columns) και 105555 εγγραφές. Στους παρακάτω πίνακες δίνεται μία σύντομη περιγραφή της κάθε μεταβλητής:

\begin{center}
    \includegraphics[width=1\textwidth]{images/variables_1.png}
\end{center}

\begin{center}
    \includegraphics[width=1\textwidth]{images/variables_2.png}
\end{center}

\newpage
\section{Data preproccesing}
Το πρώτο βήμα για την προεπεξεργασία των δεδομένων ήταν να κρατήσουμε μία γραμμή για κάθε 'ID'. Σε ένα 'ID' μπορεί να αντιστοιχούν  περισσότερες από μία γραμμές που αντιπροσωπεύουν το 
ίδιο συμβόλαιο του ίδιου πελάτη για διαφορετική χρονική περίοδο. Έτσι, για κάθε 'ID' κρατάμε την τελευταία ανανέωση του συμβολαίου, δηλαδή την γραμμή με το μεγαλύτερο χρονολογικά 'last\_renewal\_date'. 
Έπειτα, στη θέση του 'premium' υπολογίζουμε και τοποθετούμε τον μέσο όρο των 'premium' όλων των γραμμών με κοινό 'ID'.

Το δεύτερο βήμα ήταν η επεξεργασία όλων των ημερομηνιών. Ειδικότερα, οι στήλες 'Date\_birth', 'Date\_driving\_license', 'Date\_start\_contract', 'Date\_last\_renewal', \\
'Date\_next\_renewal', 'Date\_lapse' δίνονται στην μορφή ΗΗ/ΜΜ/ΕΕΕΕ. Αρχικά, για κάθε μία από αυτές τις μεταβλητές κρατήσαμε το έτος (ΕΕΕΕ) και στην συνέχεια πραγματοποιώντας τις κατάλληλες αφαιρέσεις
δημιουργήσαμε νέες στήλες στο dataset που πήραν την θέση αυτών που αναφέρθηκαν νωρίτερα. Έτσι, δημιουργήσαμε τις στήλες: 'Age' που προσδιορίζει την ηλικία του πελάτη, 'Years\_driving' που προσδιορίζει 
πόσα χρόνια οδηγεί ο πελάτης, 'Year\_on\_road' που προσδιορίζει πόσα χρόνια κυκλοφορεί το κάθε όχημα υπό την κατοχή συγκεκριμένου πελάτη, 'Policy Duration' που υποδεικνύει την διάρκεια του εκάστοτε συμβολαίου σε χρόνια και 
'Years\_on\_policy' που προσδιορίζει πόσα χρόνια ο πελάτης βρίσκεται στον ίδιο τύπο συμβολαίου. Πρέπει να σημειωθεί πως το dataset περιέχει δεδομένα μέχρι και το 2019. Για να έχουμε μια σωστή εικόνα των χρονολογιών
σε όλες αυτές τις μεταβλητές που δημιουργήσαμε, χρησιμοποιήσαμε ως σημείο αναφοράς την χρονολογία τελευταίας ανανέωσης του συμβολαίου. Για παράδειγμα η μεταβλητή 'Age' προκύπτει από την αφαίρεση:
'Age' = 'Date\_last\_renewal' - 'Date\_birth'.

Επιπλέον, δημιουργήσαμε μία ακόμη νέα στήλη με όνομα 'accidents' για να υπάρχει μια συσχέτιση μεταξύ των αριθμών των ατυχημάτων με τα χρόνια που ένας πελάτης είναι ασφαλισμένος στην εταιρεία.

Διαχειριστήκαμε την απουσία τιμών με δύο τρόπους. Στην στήλη 'Length' αντικαταστήσαμε τα κενά πεδία με τον μέσο όρο των τιμών της στήλης. Στην στήλη 'Type\_fuel' αντικαταστήσαμε τα κενά πεδία με την τιμή 
'Unknown'.

Μετά από δοκιμές διαπιστώσαμε πως κάποιες μεταβλητές του dataset δεν συνεισέφεραν καθόλου στην βελτίωση της απόδοσης και παραλείφθηκαν. Οι στήλες που χρησιμοποιήθηκαν τελικά είναι οι: 
'Seniority', 'Premium', 'Type\_risk', 'Area', 'Power', 'Second\_driver', 'Years\_on\_road', 'R\_claims\_history', 'Years\_on\_policy', 'accidents', 'Value\_vehicle, 'Age', 'Years\_driving', 'Distribution\_channel', 'N\_claims\_history',
\\ 'Cylinder\_capacity', 'Weight', 'Length', 'Type\_fuel', 'Payment', 'Contract\_year', \\'Policies\_in\_force', 'Lapse'.



\section{Εκπαίδευση μοντέλου}


\noindent \textbf{XGBoost Parameters}

(Στις παρενθέσεις βρίσκεται το συνηθισμένο εύρος τιμών, δηλαδή το εύρος που έγιναν οι δοκιμές για να πετύχουμε το μέγιστο επιθυμητό αποτέλεσμα.)

\noindent \textbf{Objective (linear):}

Το `reg:squarederror` είναι το προεπιλεγμένο και ευρέως χρησιμοποιούμενο loss function για regression tasks, παρέχοντας ένα σαφές πλαίσιο για τη βελτιστοποίηση του μοντέλου XGBoost. Συγκεκριμένα, είναι κατάλληλο για την πρόβλεψη ασφάλιστρων, καθώς στοχεύει στην επίτευξη υψηλής ακρίβειας στην πρόβλεψη της συνεχούς τιμής των ασφάλιστρων.

\noindent \textbf{Eval\_metric (`rmse`, `mae`):}

Για το eval\_metric, επιλέξαμε το `rmse` (Root Mean Squared Error) επειδή είναι ένα ευρέως αποδεκτό μέτρο για regression tasks. Αναλυτικότερα, δίνει μεγαλύτερη βαρύτητα σε μεγαλύτερες αποκλίσεις (λάθη) μεταξύ πραγματικών και προβλεπόμενων τιμών, γεγονός που μπορεί να είναι χρήσιμο στην ασφαλιστική πρόβλεψη όπου τα μεγάλα λάθη μπορεί να είναι πιο σοβαρά.

\noindent \textbf{Learning\_rate (0.001-0.3):}

Ο ρυθμός μάθησης (learning rate) επηρεάζει πόσο γρήγορα προσαρμόζεται το μοντέλο κατά την εκπαίδευση. Επιλέχθηκε η τιμή 0.01 έπειτα από δοκιμές, διότι όντας χαμηλή σχετικά τιμή επιβραδύνει τη διαδικασία μάθησης, μειώνοντας την πιθανότητα υπερβολικής προσαρμογής (overfitting).

\noindent \textbf{Max\_depth (3-10):}

Το μέγιστο βάθος των δέντρων στο μοντέλο. Ένας μεγαλύτερος αριθμός επιτρέπει πιο περίπλοκα δέντρα, αλλά μπορεί να οδηγήσει σε overfitting, επομένως επιλέχθηκε η τιμή 6.

\noindent \textbf{Min\_child\_weight (1-10):}

Ελάχιστο βάρος που πρέπει να έχει ένας κόμβος για να διαχωριστεί. Επηρεάζει την ευαισθησία του μοντέλου στις διακυμάνσεις στα δεδομένα. Ύστερα από δοκιμές προτιμήθηκε ο αριθμός βάρους 5.

\noindent \textbf{Subsample (0.5-1):}

Αναφέρεται στο ποσοστό των δειγμάτων που θα χρησιμοποιηθούν σε κάθε επανάληψη της εκπαίδευσης. Ένα ποσοστό 0.8 σημαίνει ότι το 80% των δειγμάτων θα χρησιμοποιηθεί.

\noindent \textbf{Colsample\_bytree (0.5-1):}

Το ποσοστό των χαρακτηριστικών (columns) που θα επιλεχθούν τυχαία για να κατασκευάσουν κάθε δέντρο στο μοντέλο. Ένα ποσοστό 0.8 σημαίνει ότι το 80% των χαρακτηριστικών θα χρησιμοποιηθεί σε κάθε δέντρο.

\noindent \textbf{N\_estimators (100-2000):}

Ο αριθμός των δέντρων που θα κατασκευαστούν στο μοντέλο. Δοκιμάστηκαν διάφοροι αριθμοί, συμπεριλαμβανομένων μεγαλύτερων από 1000, αλλά δεν υπήρχε μεγάλη βελτίωση στην απόδοση του μοντέλου, ενώ ο χρόνος εκπαίδευσης αυξανόταν σημαντικά. Έτσι, ως optimal θεωρήθηκε η τιμή 1000.

\noindent \textbf{Seed:}

Για να καταφέρουμε να εξασφαλίσουμε επαναληψιμότητα στα δεδομένα μας και να αφαιρεθεί η τυχαιότητα, επιλέξαμε έναν σταθερό αριθμό seed 42.

\noindent \textbf{Random forest parameters:}


\noindent \textbf{N\_estimators (100-1000):}

Επιλέχθηκε η τιμή 700, διότι παρατηρήθηκε ότι, παρόλο που η αύξηση της τιμής βελτίωνε την απόδοση του RF, ο χρόνος εκτέλεσης αυξανόταν σημαντικά λόγω της δημιουργίας πολλών δέντρων.

\noindent \textbf{Max\_depth (10-20):}

Επιλέχθηκε η τιμή 15, καθώς η αύξηση της τιμής αυξάνει τον κίνδυνο overfitting, αν και προσφέρει σημαντική βελτίωση στην απόδοση του RF.

\noindent \textbf{Min\_samples\_split (2-10):}

Επιλέχθηκε η τιμή 4, που βρίσκεται στη μέση του εύρους τιμών, για να ισορροπήσει μεταξύ της πρόληψης του overfitting και της διατήρησης της απόδοσης του μοντέλου.

\noindent \textbf{Min\_samples\_leaf (2-10):}

Επιλέχθηκε η τιμή 2, διότι παρατηρήθηκε ότι η αύξηση της τιμής αυτής παραμέτρου βελτίωνε μόνο την απόδοση στο type\_risk4, ενώ έπεφτε η συνολική απόδοση σε κάθε άλλο τύπο οχήματος.

\noindent \textbf{Max\_features (`sqrt`, `log2`, ή float 0.1-1):}

Συνήθως, σε αλγορίθμους RF προτιμάται η τιμή `sqrt`, και αυτή επιλέχθηκε και από εμάς για καλύτερη απόδοση.

\noindent \textbf{Bootstrap (True, False):}

Επιλέξαμε να έχουμε bootstrap, καθώς βελτιώνονται σημαντικά τα αποτελέσματα σε σύγκριση με την επιλογή της μη ύπαρξης bootstrap.

\noindent \textbf{N\_jobs (-1, 1, αριθμός πυρήνων):}

Με την επιλογή της τιμής -1, χρησιμοποιούμε όλους τους διαθέσιμους πυρήνες.

\noindent \textbf{Random\_state:}

Πρόκειται για τιμή τυχαιότητας όπως το seed στο XGBoost. Έχει επιλεγεί και εδώ η τιμή 42.


\section{Διαγράμματα}
Τα διαγραμματα
\includegraphics[width=1\textwidth, keepaspectratio]{images/premium.png}

\begin{figure}
    \centering
     \begin{subfigure}{0.45\linewidth}
      \includegraphics[width=\linewidth]{images/premium_risk1.png}
      \caption{Motorbikes}
      \label{fig:subfig1}
     \end{subfigure}
     \begin{subfigure}{0.45\linewidth}
      \includegraphics[width=\linewidth]{images/premium_risk2.png}
      \caption{Vans}
      \label{fig:subfig2}
      \end{subfigure}
  \vfill
       \begin{subfigure}{0.45\linewidth}
       \includegraphics[width=\linewidth]{images/premium_risk3.png}
       \caption{Passenger cars}
       \label{fig:subfig3}
        \end{subfigure}
         \begin{subfigure}{0.45\linewidth}
        \includegraphics[width=\linewidth]{images/premium_risk4.png}
        \caption{Agricultural Vehicles}
        \label{fig:subfig4}
         \end{subfigure}
  \caption{Comparison of Different Vehicle Types}
  \label{fig:subfigures4}
\end{figure}


\begin{observation}
    Στο πρώτο διάγραμμα παρουσιάζεται η κατανομή των ασφαλίστρων για όλες τις καταχωρίσεις. Τα επόμενα διαγράμματα δείχνουν την κατανομή των ασφαλίστρων για κάθε κατηγορία οχημάτων ξεχωριστά.

    Εύκολα διαπιστώνεται από το πρώτο ολικό διάγραμμα ότι οι ασφαλιστικές τιμές πάνω από 500 είναι ελάχιστες και δεν επηρεάζουν σημαντικά το τελικό αποτέλεσμα. Ωστόσο, όταν κατηγοριοποιήσαμε τα δεδομένα, παρατηρήσαμε ότι η διασπορά των τιμών στα αγροτικά οχήματα ήταν μεγαλύτερη, με αποτέλεσμα η διαφοροποίηση του μοντέλου ανά κατηγορία οχήματος να είναι απαραίτητη.
\end{observation}

\section{User Interface}
Το user interface της εφαρμογής αναπτύχθηκε με χρήση της Python και ειδικότερα του framework Kivy, καθώς και της συλλογής από γραφικά στοιχεία KivyMD.
\end{document}


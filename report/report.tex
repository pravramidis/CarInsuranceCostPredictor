%  Μεταγλώττιση με XeLaTeX

\documentclass{llncs}
\pagestyle{headings} 
%
\usepackage{makeidx}  % allows for indexgeneration
%
\usepackage{xltxtra}
\usepackage{xgreek}

\usepackage{caption}
\usepackage{subcaption}

\setsansfont{Arial}
\setmonofont{Courier New}
\setmainfont[Mapping=tex-text]{Times New Roman}
% \setmainfont[Mapping=tex-text]{GFS Didot} 
% \setsansfont{GFS Didot}
% \setmonofont{GFS Didot}

\renewcommand{\abstractname}{Περίληψη}
\renewcommand{\keywordname}{\bf Λέξεις Κλειδιά:}
\renewcommand\refname{Αναφορές}
\renewcommand\ackname{Ευχαριστίες}
\renewcommand\andname{και}
\renewcommand\corollaryname{Πόρισμα}
\renewcommand\definitionname{Ορισμός}
\renewcommand\examplename{Παράδειγμα}
\renewcommand\exercisename{Άσκηση}
\renewcommand\figurename{Εικ.}
\renewcommand\lemmaname{Λήμμα}
\renewcommand\proofname{Απόδειξη}
\renewcommand\propositionname{Πρόταση}
\renewcommand\solutionname{Λύση}
\renewcommand\tablename{Πίνακας}
\renewcommand\theoremname{Θεώρημα}

\begin{document}
%
\title{\Huge Προγραμματιστική Εργασία Πρόβλεψη κόστους ασφάλισης οχημάτων}
%
%
\author{\Large Χαρά Τσίρκα \and Πρόδρομος Αβραμίδης \and Γεώργιος Γεροντίδης}
%
%
%
\institute{\email{\{ctsirka, pavramidis, ggerontidis\}@e-ce.uth.gr}\\
8$^{ο}$ εξάμηνο\\
\begin{center}
    \vspace{1cm}
    \includegraphics[width=0.4\textwidth]{uthlogo.png}
    \vspace{1cm}
\end{center}
\Large Τμήμα Ηλεκτρολόγων Μηχανικών \& Μηχανικών Υπολογιστών\\
Πανεπιστήμιο Θεσσαλίας, Βόλος\\
\vspace{1cm}
{\bf \Large Εξόρυξη Δεδομένων 2023-24}\\
\Large Διδάσκον: Μ.Βασιλακόπουλος\\
\vspace{1cm}
Μάιος 2024}

\maketitle



\begin{abstract}
    Η εργασία μας επικεντρώνεται στην πρόβλεψη του κόστους ασφάλισης μηχανοκίνητων οχημάτων. Η ανάλυση αυτή αποτελεί ένα κρίσιμο ζήτημα στον τομέα της ασφάλισης, καθώς επιτρέπει στους ασφαλιστές να προσδιορίζουν με μεγαλύτερη ακρίβεια τα ασφαλιστικά ασφάλιστρα, λαμβάνοντας υπόψη διάφορους παράγοντες που επηρεάζουν το κόστος.
    Η διαδικασία της εργασίας ξεκινά με την προ-επεξεργασία των δεδομένων, κατά την οποία πραγματοποιήθηκε εξερευνητική ανάλυση (exploratory analysis) για τον προσδιορισμό των κριτηρίων διαχωρισμού των δεδομένων. Κατά τη διάρκεια αυτής της ανάλυσης, μετρήθηκε ο βαθμός επίδρασης κάθε χαρακτηριστικού (feature) του συνόλου δεδομένων στα αποτελέσματα. Με τη βοήθεια διαγραμμάτων, καταφέραμε να επιλέξουμε τον κατάλληλο διαχωρισμό των δεδομένων για περαιτέρω ανάλυση.
    Στη συνέχεια, η εργασία θα προχωρήσει στη δημιουργία και αξιολόγηση των μοντέλων πρόβλεψης, λαμβάνοντας υπόψη την είσοδο των χρηστών, ενώ θα ακολουθήσει η οπτικοποίηση και η αξιολόγηση των αποτελεσμάτων.
\end{abstract}


\includegraphics[width=0.7\textwidth, keepaspectratio]{images/premium.png}


\begin{figure}
    \centering
     \begin{subfigure}{0.45\linewidth}
      \includegraphics[width=\linewidth]{images/premium_risk1.png}
      \caption{Motorbikes}
      \label{fig:subfig1}
     \end{subfigure}
     \begin{subfigure}{0.45\linewidth}
      \includegraphics[width=\linewidth]{images/premium_risk2.png}
      \caption{Vans}
      \label{fig:subfig2}
      \end{subfigure}
  \vfill
       \begin{subfigure}{0.45\linewidth}
       \includegraphics[width=\linewidth]{images/premium_risk3.png}
       \caption{Passenger cars}
       \label{fig:subfig3}
        \end{subfigure}
         \begin{subfigure}{0.45\linewidth}
        \includegraphics[width=\linewidth]{images/premium_risk4.png}
        \caption{Agricultural Vehicles}
        \label{fig:subfig4}
         \end{subfigure}
  \caption{Comparison of Different Vehicle Types}
  \label{fig:subfigures4}
\end{figure}

\end{document}

